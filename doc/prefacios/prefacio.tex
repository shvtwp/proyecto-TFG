\thispagestyle{empty}

\begin{center}
{\large\bfseries Heráldica para todas \\ Subtítulo }\\
\end{center}
\begin{center}
Lucía González Sánchez\\
\end{center}

%\vspace{0.7cm}

\vspace{0.5cm}
\noindent\textbf{Palabras clave}: \textit{heráldica digital, blasones, base de datos, consulta de escudos, 
software libre, Python, Flask, SQLModel, PostgreSQL, Git, GitHub, metodologías ágiles, desarrollo web}
\vspace{0.7cm}

\noindent\textbf{Resumen}\\
El proyecto consiste en el desarrollo de una aplicación web dedicada a la digitalización, gestión y 
consulta de blasones dentro del ámbito de la heráldica. Utilizando tecnologías de software libre como 
Python, Flask y SQLModel, la aplicación permite almacenar, consultar y filtrar escudos de armas según 
diferentes criterios heráldicos, como esmaltes, muebles, piezas o adornos exteriores.

El sistema se apoya en una base de datos estructurada que facilita la organización y el acceso coherente 
a la información heráldica, sentando las bases para futuras ampliaciones y posibles aplicaciones de 
inteligencia artificial. El desarrollo se llevó a cabo siguiendo metodologías ágiles y empleando Git y 
GitHub para el control de versiones y la gestión colaborativa del proyecto, priorizando la modularidad, 
la escalabilidad y la claridad del código.

Este trabajo ofrece una contribución significativa a la digitalización del patrimonio heráldico, al 
proporcionar una herramienta moderna, abierta y extensible para la consulta y el estudio sistemático de 
los escudos de armas.

\cleardoublepage

\begin{center}
	Lucía González Sánchez\\
\end{center}
\vspace{0.5cm}
\noindent\textbf{Keywords}: \textit{digital heraldry, coats of arms, database, coat of arms catalog, 
free software, Python, Flask, SQLModel, PostgreSQL, Git, GitHub, agile methodologies, web development}, \textit{floss}
\vspace{0.7cm}

\noindent\textbf{Abstract}\\
This project presents the development of a web application dedicated to the digitization, management, and 
retrieval of coats of arms within the field of heraldry. Using open-source technologies such as Python, 
Flask, and SQLModel, the application enables the storage, querying, and filtering of heraldic records 
according to various criteria, including tinctures, charges, ordinaries, and external ornaments.

The system relies on a structured database that ensures consistent organization and access to heraldic 
information, establishing a foundation for future extensions and potential applications of artificial 
intelligence. The development process followed agile methodologies and employed Git and GitHub for version
control and collaborative project management, emphasizing modularity, scalability, and code clarity.

This work contributes to the digital preservation of heraldic heritage by providing a modern, open, and 
extensible tool for the consultation and systematic study of coats of arms.

\cleardoublepage

\thispagestyle{empty}

\noindent\rule[-1ex]{\textwidth}{2pt}\\[4.5ex]

D. \textbf{Juan Julián Merelo Guervós}, Profesor(a) del\dots

\vspace{0.5cm}

\textbf{Informo:}

\vspace{0.5cm}

Que el presente trabajo, titulado \textit{\textbf{Heráldica para todas}},
ha sido realizado bajo mi supervisión por \textbf{Lucía González Sánchez}, y autorizo la defensa de dicho trabajo ante el tribunal
que corresponda.

\vspace{0.5cm}

Y para que conste, expiden y firman el presente informe en Granada a \today.

\vspace{1cm}

\textbf{El director: }

\vspace{5cm}

\noindent \textbf{Juan Julián Merelo Guervós}

\chapter*{Agradecimientos}




