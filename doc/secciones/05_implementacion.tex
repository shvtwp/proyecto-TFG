\chapter{Implementación}

La implementación del software se ha dividido en hitos. Estos han sido definidos en GitHub
y cada uno de ellos contiene un grupo de \textit{issues} que se corresponden con las distintas
mejoras que se han ido incorporando al software a lo largo de su desarrollo.

Cada milestone representa un bloque de trabajo con un objetivo definido, lo que permite seguir
un enfoque ágil, enfocado en construir progresivamente un prototipo sólido en lugar de intentar
desarrollar todo el sistema de una vez. 

\section{Milestone 1: Elección del lenguaje de programación}
En este primer hito se decidió el lenguaje de programación a utilizar para el desarrollo del software,
siendo Python la opción elegida. Esta elección se justificó y documentó debidamente en el
\hyperref[sec:lenguaje-programacion]{anterior capítulo}.