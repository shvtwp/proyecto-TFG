\chapter{Planificación}

\section{Metodología utilizada}

\subsection{Herramientas utilizadas}

\subsubsection{Memoria}
Para la realización de esta memoria, se ha establecido como requisito mínimo que
el documento esté libre de faltas de ortografía y compile correctamente. Partiendo de que 
está escrito en \LaTeX, se han buscado herramientas que permitan la automatización de estas 
comprobaciones. Para ello se han estudiado dos tipos de herramientas: las destinadas a la
corrección gramatical y de estilo, y las empleadas para compilar la memoria.
\\

Entre las \textbf{herramientas para la corrección ortográfica}, han sido analizadas las siguientes
opciones:

\begin{itemize}
    \item \textbf{Textidote}: es una solución de código abierto diseñada específicamente para
    \LaTeX\, lo que evita falsos positivos por malinterpretar los comandos. Requiere Java.
    \item \textbf{LanguageTool}: esta opción de código abierto es más avanzada que la anterior 
    debido a que usa redes neuronales para comprobar la gramática y el estilo. 
    Sin embargo, no está diseñada específicamente para \LaTeX, lo que hace necesario 
    invertir tiempo extra en eliminar los comandos antes de realizar las verificaciones.
    Tiene una versión gratuita y una versión prémium con funcionalidades avanzadas.
    \item \textbf{GNU Aspell}: es el corrector ortográfico estándar de GNU, aunque compila para
    otros sistemas operativos. Es de código abierto, no corrige gramática ni estilo, solo
    ortografía, y no está diseñado específicamente para \LaTeX.
\end{itemize}

Después de la comparativa, la opción elegida ha sido \textbf{Textidote}, porque a pesar de 
no ser tan avanzada en la corrección, supone un ahorro considerable de tiempo el hecho de no 
tener que eliminar los comandos de \LaTeX\ al verificar la gramática y el estilo, como sucede
con las otras dos alternativas analizadas.
\\

Con respecto a las \textbf{herramientas para la compilación de la memoria}, se han estudiado:

\begin{itemize}
    \item \textbf{TeX Live}: es la distribución por defecto de \LaTeX\ para Linux. Sin embargo,
    la instalación puede ser pesada y excesiva si únicamente se necesita compilar documentos
    sencillos.
    \item \textbf{Overleaf}: es un editor \LaTeX\ basado en la nube, por lo que no requiere 
    instalación previa. Es ideal para trabajar de manera colaborativa al permitir
    a varios usuarios trabajar en el mismo documento y ver los cambios a tiempo real.
    No obstante, tiene limitaciones en cuanto a capacidad, que pueden solventarse con
    la versión de pago.
    \item \textbf{TinyTex}: es una versión basada en Tex Live, pero mucho más ligera. Está orientada
    sobre todo a usuarios de R, aunque es posible instalar manualmente los paquetes 
    específicos que sean necesarios, lo que la convierte en una buena opción en un contexto 
    en el que los recursos están limitados.
\end{itemize}

Dado a la complejidad añadida de instalar manualmente los paquetes específicos necesarios
para el proyecto y que los recursos no son limitantes en este caso, la herramienta elegida
es \textbf{TeX Live}. Se descarta \textbf{Overleaf} debido a la limitación de la versión gratuita,
además de que no será necesario un entorno de trabajo colaborativo.

\section{Temporización}

\section{Seguimiento del desarrollo}
