\chapter{Planificación}
En esta sección se presenta la planificación del proyecto. Se describen tanto la
metodología empleada como las herramientas elegidas para llevar a cabo el desarrollo,
asegurando la calidad del producto y su entrega continua. Además, se describen las historias
de usuario que exponen las necesidades que este trabajo deberá satisfacer, así como los
primeros hitos que se buscará alcanzar.

\section{Metodología utilizada}
Para la planificación y el desarrollo del proyecto se aplicará un enfoque ágil,
priorizando la entrega continua de valor y la flexibilidad ante los cambios.
En lugar de plantear el desarrollo desde el principio de una manera rígida y detallada,
se definirán hitos, con tareas y objetivos a corto plazo, con entregas frecuentes
y revisiones periódicas para poder detectar y corregir errores a tiempo.

Dentro de este enfoque ágil, se usará una adaptación de SCRUM como marco
de trabajo. Esta metodología ágil estructura el trabajo en ciclos cortos (\textit{sprints}),
de los cuales resultan entregas incrementales del producto. 

Concretamente, cada iteración de este proyecto estará marcada por un hito (\textit{milestone}),
cada uno compuesto por una serie de tareas (\textit{issues}) que se irán completando a lo largo
del desarrollo.

Al final de cada hito, se realizará una comprobación del progreso y se evaluará si cumple con los 
objetivos establecidos, y en caso contrario, se ajustarán los requisitos y se planificarán 
los siguientes hitos.

\subsection{Historias de usuario}
Como punto de partida para la planificación del proyecto, se han definido una seria de 
historias de usuario que reflejan las necesidades de los usuarios que el proyecto final
debe satisfacer. Estas historias de usuario servirán como base para la definición de los hitos.
Las historias de usuario definidas son las siguientes:
\begin{itemize}
    \item \textbf{HU1: Estudiante de Historia del Arte - }Un estudiante de Historia del 
    Arte está paseando por Roma y se encuentra con una obra arquitectónica en cuya fachada
    hay un escudo heráldico. Se pregunta de quién será y qué representa. Sigue caminando 
    y se encuentra con más escudos, lo que le hace pensar en lo oportuna que sería una 
    plataforma que le ayude a identificarlos y obtener información sobre ellos de manera 
    sencilla.
    \item \textbf{HU2: Investigador en heráldica - }Un investigador encuentra tedioso y
    complicado analizar y comprobar escudos en fuentes analógicas, ya que hacerlo manualmente
    de esta manera lo convierte en un proceso lento y propenso a errores.
\end{itemize}

\subsection{Objetivos iniciales}
A partir de las anteriores historias de usuario, se puede establecer el primer
producto mínimamente viable (o primer milestone):
\begin{enumerate}
    \item \textbf{Milestone 0: Infraestructura del proyecto - }Este primer hito tiene como 
    objetivo crear la infraestructura del proyecto, incluyendo la creación del repositorio
    en GitHub, la planificación de los hitos y tareas iniciales, y las herramientas para automatizar
    la compilación y corrección ortográfica y gramatical de la memoria.
    \item \textbf{Milestone 1: Modelo del problema - }Este milestone tiene como objetivo
    elaborar una base sólida mediante la creación de una representación básica y estructurada
    de los escudos, lo que permitirá almacenar y organizar sus atributos clave.
\end{enumerate}

A medida que se avance en este milestone, se establecerá el siguiente paso de manera iterativa,
ajustándose a los resultados obtenidos y con la finalidad de ofrecer valor continuo al usuario.

\subsection{Calidad y entrega continua}
Con el fin de garantizar la calidad del producto y minimizar los errores, se aplicará
un enfoque de control de calidad continuo. Esto implica la implementación de pruebas unitarias
y de integración, así como la revisión y verificación de la documentación generada.
Implementar un enfoque de entrega continua (CD/CI) permitirá automatizar el proceso.


\subsection{Herramientas utilizadas}
Para dar soporte al flujo de trabajo, se utilizarán las siguientes herramientas:
\begin{itemize}
    \item \textbf{Git}: se empleará como sistema de control de versiones para gestionar el código,
    realizar un seguimiento de los cambios, crear ramas de desarrollo y mantener un historial
    claro y ordenado del proyecto.
    \item \textbf{GitHub}: se utilizará como plataforma centralizada para alojar el repositorio
    de código, facilitando la gestión de los \textit{issues} y \textit{milestones}, además de la
    configuración de la integración continua (CI/CD) mediante \textit{GitHub Actions.}
\end{itemize}

\subsubsection{Memoria}
Para la realización de esta memoria, se ha establecido como requisito mínimo que
el documento esté libre de faltas de ortografía y compile correctamente. Partiendo de que 
está escrito en \LaTeX, se han buscado herramientas que permitan la automatización de estas 
comprobaciones. Para ello se han estudiado dos tipos de herramientas: las destinadas a la
corrección gramatical y de estilo, y las empleadas para compilar la memoria.
\\

Entre las \textbf{herramientas para la corrección ortográfica}, han sido analizadas las siguientes
opciones:

\begin{itemize}
    \item \textbf{Textidote}: es una solución de código abierto diseñada específicamente para
    \LaTeX\, lo que evita falsos positivos por malinterpretar los comandos. Requiere Java.
    \item \textbf{LanguageTool}: esta opción de código abierto es más avanzada que la anterior 
    debido a que usa redes neuronales para comprobar la gramática y el estilo. 
    Sin embargo, no está diseñada específicamente para \LaTeX, lo que hace necesario 
    invertir tiempo extra en eliminar los comandos antes de realizar las verificaciones.
    Tiene una versión gratuita y una versión prémium con funcionalidades avanzadas.
    \item \textbf{GNU Aspell}: es el corrector ortográfico estándar de GNU, aunque compila para
    otros sistemas operativos. Es de código abierto, no corrige gramática ni estilo, solo
    ortografía, y no está diseñado específicamente para \LaTeX.
\end{itemize}

Después de la comparativa, la opción elegida ha sido \textbf{Textidote}, porque a pesar de 
no ser tan avanzada en la corrección, supone un ahorro considerable de tiempo el hecho de no 
tener que eliminar los comandos de \LaTeX\ al verificar la gramática y el estilo, como sucede
con las otras dos alternativas analizadas.
\\

Con respecto a las \textbf{herramientas para la compilación de la memoria}, se han estudiado:

\begin{itemize}
    \item \textbf{TeX Live}: es la distribución por defecto de \LaTeX\ para Linux. Sin embargo,
    la instalación puede ser pesada y excesiva si únicamente se necesita compilar documentos
    sencillos.
    \item \textbf{Overleaf}: es un editor \LaTeX\ basado en la nube, por lo que no requiere 
    instalación previa. Es ideal para trabajar de manera colaborativa al permitir
    a varios usuarios trabajar en el mismo documento y ver los cambios a tiempo real.
    No obstante, tiene limitaciones en cuanto a capacidad, que pueden solventarse con
    la versión de pago.
    \item \textbf{TinyTex}: es una versión basada en Tex Live, pero mucho más ligera. Está orientada
    sobre todo a usuarios de R, aunque es posible instalar manualmente los paquetes 
    específicos que sean necesarios, lo que la convierte en una buena opción en un contexto 
    en el que los recursos están limitados.
\end{itemize}

Dado a la complejidad añadida de instalar manualmente los paquetes específicos necesarios
para el proyecto y que los recursos no son limitantes en este caso, la herramienta elegida
es \textbf{TeX Live}. Se descarta \textbf{Overleaf} debido a la limitación de la versión gratuita,
además de que no será necesario un entorno de trabajo colaborativo.

\section{Temporización}
La temporización se establecerá en función de los hitos y las tareas necesarias para 
alcanzar los objetivos propuestos.