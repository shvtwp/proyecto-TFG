\chapter{Estado del arte}

En este capítulo se realiza un análisis de las herramientas y tecnologías
existentes que permiten analizar y/o clasificar de manera digital la heráldica, con el fin
de determinar en qué medida responden a las necesidades del cliente.
El objetivo de este capítulo es identificar qué aspectos están ya resueltos,
cuáles son sus limitaciones y qué vacíos justifican la propuesta del proyecto.
Además, servirá como base para la toma de decisiones tecnológicas en fases posteriores.

\section{Enfoque del análisis}
El objetivo de este capítulo es estudiar cómo se han abordado hasta ahora el análisis y
clasificación de escudos de armas en el ámbito digital, y evaluar en qué medida estas soluciones 
responden a las necesidades del proyecto y sus futuros usuarios.
Tal y como se indica en la descripción del problema, se compararán los siguientes aspectos 
relacionados con el análisis y la clasificación de escudos de armas:

\begin{itemize}
    \item \textbf{Representación visual}: Es un aspecto clave para usuarios que desean identificar
    los escudos de manera intuitiva, por ejemplo, en la fachada de un edificio. Se examinarán las 
    herramientas que permiten generar imágenes de escudos a partir de descripciones textuales o 
    mediante interfaces gráficas.
    \item \textbf{Estructuración semántica}: Se revisarán las ontologías y esquemas que
    organizan la información heráldica de manera estructurada y reutilizable para facilitar
    su análisis y procesamiento automático.
    \item \textbf{Usabilidad y accesibilidad}: Se evaluará la facilidad de uso de las herramientas
    para usuarios no expertos, además de su utilidad para tareas de análisis o clasificación.
    \item \textbf{Tipo de heráldica}: El tipo de heráldica que cubren (nacional, eclesiástica, etc.).
\end{itemize}

Este análisis hará posible la identificación de las limitaciones y las áreas que aún no han sido
cubiertas, lo que justificará el desarrollo de este proyecto.

\section{Tecnologías actuales en el ámbito de la heráldica}
En esta sección se analizan las tecnologías y herramientas existentes aplicadas a la representación
digital de la heráldica, con el fin de detectar las limitaciones actuales y las oportunidades
de mejora para satisfacer al cliente.

Por un lado, se analizarán las herramientas de software que ya representan escudos heráldicos,
y por otro, se explorarán propuestas de organización de la información más estructuradas
(como ontologías o esquemas semánticos). Estos recursos permiten definir con precisión
qué elementos componen un escudo y cómo se relacionan entre sí, lo cual resulta útil para
quienes deseen identificar un escudo a partir de su descripción textual.

\subsection{Sistemas de representación heráldica}
En el contexto del modelado digital de blasones, existen diversas herramientas y repositorios
que ya trabajan con representaciones heráldicas. Algunos sistemas se centran en la generación 
visual de escudos, otros en la interpretación computacional del lenguaje heráldico, y otros 
simplemente recopilan grandes volúmenes de información sin un tratamiento semántico explícito. 
A continuación, se analizan algunos ejemplos desde los criterios previamente mencionados:

\begin{itemize}
    \item \textbf{\href{https://drawshield.net/index.html}{DrawShield}} es una herramienta online que permite generar escudos de armas
    a partir de blasones escritos en inglés. Utiliza un lenguaje formalizado cercano al blasonado
    tradicional, y produce imágenes SVG, facilitando su utilización como recurso visual. Aunque está 
    centrada en la \hyperref[sec:heraldica]{heráldica anglosajona} y no es completamente de código abierto, su 
    uso es gratuito y posee suficiente documentación técnica pública para entender su 
    funcionamiento. Puede resultar útil para comprender cómo formalizar el lenguaje heráldico
    y traducirse automáticamente a representaciones visuales, aunque su enfoque no se adapta
    completamente a la heráldica española o eclesiástica.
    \item \textbf{\href{https://coamaker.com/}{Coat of Arms Maker}} son un conjunto de herramientas gráficas que permiten
    la creación de escudos desde una interfaz visual. Tiene fines recreativos y no sigue
    necesariamente las reglas formales del blasonado ni genera datos estructurados. Su alcance
    desde el punto de vista heráldico es bastante superficial. No tiene mucha utilidad real
    en el proyecto, pero puede servir como referencia para entender cómo se pueden representar
    visualmente los elementos heráldicos de forma intuitiva.
    \item \textbf{\href{https://web.meson.org/pyBlazon/}{pyBlazon}} es una librería en Python orientada a procesar blasones escritos en
    inglés, con el objetivo de analizar su estructura. Aún está en desarrollo, por lo que tiene
    una cobertura del vocabulario heráldico limitada, centrada además en escudos ingleses.
    Reconoce los elementos básicos del blasonado, como figuras, esmaltes y particiones.
    Aunque no genera representaciones visuales por sí sola, puede ser útil como un punto de partida
    para estudiar cómo se puede interpretar el lenguaje heráldico de modo automático. Habría que
    adaptarla para que pueda trabajar con la heráldica española y eclesiástica.
    \item \textbf{Wikipedia y Wikimedia Commons:} si bien no son herramientas en sentido estricto,
    estas plataformas representan una gran fuente de escudos digitalizados con descripciones
    textuales (blasones) en múltiples idiomas. Aunque la representación no sigue un sistema formal 
    estructurado ni está pensada para el proceso automático, su volumen y variedad de datos
    disponibles las convierte en candidatas para procesos de extracción de conocimiento o análisis.
\end{itemize}

En general, estos sistemas tienen distintas finalidades y niveles de detalle. Algunos se centran
en la representación visual de escudos, otros en la interpretación del lenguaje heráldico, y
otros simplemente recopilan información sin un tratamiento semántico explícito, no obstante ninguno
de ellos ofrece un modelo completo que permita el análisis profundo y la integración de la 
heráldica en entornos digitales inteligentes.

\subsection{Ontologías y esquemas semánticos en heráldica}
En el ámbito de la representación formal del conocimiento heráldico, las ontologías y los esquemas 
semánticos han cobrado importancia como herramientas que permiten estructurar la información de
forma que sea comprensible tanto por humanos como por máquinas. Estas herramientas buscan representar
los elementos de un escudo (tales como figuras, particiones o esmaltes) y sus relaciones, de manera
lógica y reutilizable, lo que resulta especialmente útil en aplicaciones digitales como bases de 
datos, sistemas de recuperación de información o inteligencia artificial.

\begin{itemize}
    \item \textbf{\href{https://finto.fi/en/ontology/heraldica/}{HERO – Ontology of Heraldry}} desarrollada por los Archivos Nacionales de Finlandia, 
    esta ontología proporciona un vocabulario estructurado destinado a describir conceptos heráldicos.
    Se utiliza en iniciativas como Europeana Heraldica con el fin de mejorar la búsqueda y la organización
    de bases de datos de escudos. Además, unifica los términos heráldicos en varios idiomas, aunque se
    centra en la tradición nórdica y europea central. Puede servir como referencia para estructurar un
    vocabulario propio adaptado.
    \item \textbf{\href{https://digitalheraldry.org/digital-heraldry-ontology/heraldry/documentation/index-en.html}{Digital Heraldry Ontology (Humboldt-Universität zu Berlin)}} esta ontología surge del
    proyecto “Coats of Arms in Practice” y tiene como objetivo representar digitalmente escudos de armas
    siguiendo las reglas del lenguaje heráldico. Organiza la información de los componentes de un blasón,
    lo que facilita su análisis, clasificación y generación automática. Tiene gran potencial, aunque está
    redactada en inglés y se centra principalmente en la heráldica alemana, lo que limita su aplicabilidad
    a la heráldica española y eclesiástica. Aun así, puede servir como base para desarrollar un modelo
    semántico adaptado a las necesidades del proyecto.
\end{itemize}

Ambas iniciativas muestran enfoques complementarios: mientras HERO se centra más en organizar y
definir términos heráldicos, la Digital Heraldry Ontology se enfoca en representar y analizar
formalmente los escudos. No obstante, ambas presentan limitaciones para la \hyperref[sec:heraldica]{heráldica española y
eclesiástica}, por lo que podrían servir como punto de partida para la creación de un modelo semántico
más completo y adaptado a las necesidades específicas de este proyecto.

\section{Conclusión del capítulo}
Las herramientas actuales presentes en el ámbito de la heráldica digital abarcan un amplio
rango de enfoques, aunque incompletos. Por un lado, las herramientas orientadas a la representación
visual o recopilación de datos tratan de forma limitada la estructura semántica. Por otro lado,
las ontologías y esquemas semánticos ofrecen una mejor organización para la interpretación
automática, pero aún no están suficientemente integradas en las herramientas de uso general y
suelen estar enfocadas a ámbitos geográficos concretos. 

En particular, la heráldica española y eclesiástica no se encuentran suficientemente representadas 
en ellas, ni desde el punto de vista visual ni semántico, lo que evidencia la falta de soluciones 
que respondan a las necesidades identificadas en las historias de usuario.

Estas limitaciones no solo representan un reto técnico, sino que afectan directamente a los 
usuarios finales. Tanto los estudiantes interesados en identificar escudos en su contexto patrimonial 
como los investigadores que trabajan con fuentes dispersas y poco accesibles se enfrentan a 
dificultades concretas que actualmente no están resueltas por las herramientas existentes.

Estas carencias exponen la necesidad de un enfoque integral que combine representación
visual, estructuración semántica y accesibilidad. Este proyecto pretende cubrir ese vacío,
aportando una solución digital que responda a problemas reales.