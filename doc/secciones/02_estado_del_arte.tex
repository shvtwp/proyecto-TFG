\chapter{Estado del arte}

En este capítulo se realiza un análisis de las herramientas y tecnologías
disponibles que pueden ser utilizadas en el proyecto. Dado que el trabajo se 
encuentra en una fase inicial, aún no se pueden definir completamente todos
los requisitos funcionales. Sin embargo, a partir de las necesidades
generales identificadas, se puede realizar un estudio del estado actual
de las soluciones disponibles.

El objetivo de este capítulo es servir como base para la toma de 
decisiones tecnológicas en fases posteriores del proyecto.

\section{Enfoque del análisis}
Como no es posible definir todos los requisitos funcionales del proyecto
desde el inicio, se establecen unos criteros básicos y flexibles para
poder guiar el análisis de las herramientas y tecnologías:

\begin{itemize}
    \item \textbf{Flexibilidad}: Se desean herramientas que no limiten el desarrollo
    futuro del proyecto, permitiendo adaptaciones y ampliaciones sin aumentar la dificultad.
    \item \textbf{Comunidad y soporte}: Se priorizarán aquellas con una comunidad activa
    y buena documentación.
    \item \textbf{Compatibilidad}: Se deberán integrar fácilmente con otras tecnologías
    que se vayan a utilizar.
    \item \textbf{Licencia}: Se preferirán herramientas de código abierto debido al
    contexto académico del proyecto.
    \item \textbf{Rendimiento}: Se valorará su capacidad para manejar grandes volúmenes
    de datos o tareas complejas.
\end{itemize}

Estas directrices servirán para evaluar las diferentes opciones y seleccionar las más 
adecuadas para el proyecto.

\section{Tecnologías actuales en el ámbito de la heráldica}
En esta sección se revisan las tecnologías y herramientas existentes que pueden ser
aplicadas al ámbito de la heráldica, con el fin de identificar tanto sus puntos fuertes
como las necesidades que aún no están cubiertas.

Por un lado, se analizarán las herramientas de software que ya representan escudos heráldicos,
y por otro, se explorarán propuestas de organización de la información más estructuradas
(como ontologías o esquemas semánticos). Estos recursos permiten definir con precisión
qué elementos componen un escudo y cómo se relacionan entre sí, lo cual resulta útil si
en fases futuras se desea automatizar su interpretación o generar documentación técnica.

\subsection{Sistemas de representación heráldica}
En el contexto del modelado digital de blasones, existen diversas herramientas y repositorios
que ya trabajan con representaciones heráldicas de forma más o menos estructurada. Algunos 
sistemas se centran en la generación visual de escudos, otros en la interpretación computacional
del lenguaje heráldico, y otros simplemente recopilan grandes volúmenes de información sin un
tratamiento semántico explícito. A continuación, se analizan algunos ejemplos:

\begin{itemize}
    \item \textbf{DrawShield:} es una herramienta online que permite generar escudos de armas
    a partir de blasones escritos en inglés. Utiliza un lenguaje formalizado cercano al blasonado
    tradicional, y produce imágenes SVG, facilitando su uso como recurso visual. Aunque está 
    centrada en la heráldica anglosajona y no es completamente de código abierto, su 
    uso es gratuito y posee suficiente documentación técnica pública para permite entender su 
    funcionamiento. Puede resultar útil como apoyo visual o para explorar la traducción del
    texto heráldico a imágenes.
    \item \textbf{Coat of Arms Maker:} son un conjunto de herramientas gráficas que permiten
    la creación de escudos desde una interfaz visual. Tiene fines recreativos y no sigue
    necesariamente las reglas formales del blasonado ni genera datos estructurados. Su alcance
    desde el punto de vista heráldico es bastante superficial.
    \item \textbf{pyBlazon:} es una librería en Python orientada a procesar blasones escritos en
    inglés, con el objetivo de analizar su estructura. Aún está en desarrollo, por lo que tiene
    una cobertura del vocabulario heráldico limitada, centrada además en escudos ingleses.
    Reconoce los elementos básicos del blasonado, como figuras, esmaltes y particiones.
    No genera representaciones visuales por sí sola, pero puede ser útil como un punto de partida
    para estudiar cómo se puede interpretar el lenguaje heráldico de forma automática.
    \item \textbf{Wikipedia y Wikimedia Commons:} si bien no son herramientas en sentido estricto,
    estas plataformas representan una gran fuente de escudos digitalizados con descripciones
    textuales (blasones) en múltiples idiomas. Aunque la representación no sigue un sistema formal 
    estructurado ni está pensada para el proceso automático, su volumen y variedad de datos
    disponibles las convierte en candidatas para procesos de extracción de conocimiento o análisis.
\end{itemize}

En general, estos sistemas tienen distintas finalidades y niveles de detalle. Algunos se centran
en la representación visual de escudos, otros en la interpretación del lenguaje heráldico, y
otros simplemente recopilan información sin un tratamiento semántico explícito, pero ninguno
de ellos ofrece un modelo completo que permita el análisis profundo y la integración de la 
heráldica en entornos digitales inteligentes.

\subsection{Ontologías y esquemas semánticos en heráldica}
En el ámbito de la representación formal del conocimiento heráldico, las ontologías y los esquemas 
semánticos han cobrado importancia como herramientas que permiten estructurar la información de
forma que sea comprensible tanto por humanos como por máquinas. Estas herramientas buscan representar
los elementos de un escudo (tales como figuras, particiones o esmaltes) y sus relaciones, de forma
lógica y reutilizable, lo que resulta especialmente útil en aplicaciones digitales como bases de 
datos, sistemas de recuperación de información o inteligencia artificial.

\begin{itemize}
    \item \textbf{HERO – Ontology of Heraldry:} desarrollada por los Archivos Nacionales de Finlandia, 
    esta ontología proporciona un vocabulario estructurado destinado a describir conceptos heráldicos.
    Se utiliza en iniciativas como Europeana Heraldica con el fin de mejorar la búsqueda y la organización
    de bases de datos de escudos. Además, unifica los términos heráldicos en varios idiomas.
    \item \textbf{Digital Heraldry Ontology (Humboldt-Universität zu Berlin):} esta ontología surge del
    proyecto “Coats of Arms in Practice” y tiene como objetivo representar digitalmente escudos de armas
    siguiendo las reglas del lenguaje heráldico. Organiza la información de los componentes de un blasón,
    lo que facilita su análisis, clasificación y generación automática.
\end{itemize}

Ambas iniciativas muestran enfoques complementarios: mientras HERO se centra más en organizar y
definir términos heráldicos, la Digital Heraldry Ontology se enfoca en representar y analizar
formalmente los escudos, por lo que ambas podrían servir como referencia para crear un modelo
semántico de blasones en proyectos digitales. 

\section{Conclusión del capítulo}
Las herramientas actuales presentes en el ámbito de la heráldica digital abarcan un amplio
rango de enfoques. Por un lado, las herramientas orientadas a la representación visual o
recopilación de datos tratan de forma limitada la estructura semántica. Por otro lado,
las ontologías y esquemas semánticos ofrecen una mejor organización para la interpretación
automática, pero aún no están suficientemente integradas en las herramientas de uso general y
suelen estar enfocadas a ámbitos geográficos concretos. En particular, la heráldica española y papal
no se encuentra suficientemente representada en ellas, lo que abre un espacio para el desarrollo
de un modelo semántico que permita un análisis de escudos más preciso y automático.