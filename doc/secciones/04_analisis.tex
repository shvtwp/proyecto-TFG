\chapter{Análisis del problema}
Este capítulo tiene como objetivo analizar en profundidad el problema al que se pretende 
dar solución, abordando tanto los aspectos conceptuales del dominio como las implicaciones 
técnicas y funcionales del sistema desde una perspectiva inicial. No se busca definir una
solución, sino comprender el contexto del problema y acotarlo de manera realista.

Se analizarán los riesgos técnicos asociados y se esbozarán posibles vías de evolución futura. 
Finalmente, se elegirán el lenguaje de programación y las herramientas que servirán de base 
para las decisiones de implementación que se describirán en el capítulo siguiente.

\section{Análisis de riesgos}
Antes de comenzar el desarrollo se identifican los siguientes riesgos:

\begin{itemize}
    \item \textbf{Ambigüedad conceptual: } ciertos términos heráldicos pueden tener distintos
    significados según la tradición.
    \item \textbf{Falta de datos estructurados: } no existen fuentes públicas completas con escudos
    de forma estandarizada y digital, lo que puede dificultar la extracción automática de información.
\end{itemize}

\section{Evolución futura}
Aunque no forma parte del alcance inicial y siguiendo la metodología el desarrollo es incremental, 
es importante considerar desde un principio ciertos aspectos que podrían evolucionar en el futuro:

\begin{itemize}
    \item \textbf{Integración de inteligencia artificial: } para identificar visualmente los escudos.
    Puede ser un objetivo a largo plazo relacionado con la HU1.
    \item \textbf{Ampliación del sistema: } a otras tradiciones heráldicas o idiomas.
\end{itemize}

Contemplar estas posibilidades desde el inicio permite que la selección de herramientas y tecnologías
no limite o complique el crecimiento futuro del sistema.