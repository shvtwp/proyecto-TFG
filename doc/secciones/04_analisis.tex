\chapter{Análisis del problema}
Este capítulo tiene como objetivo analizar en profundidad el problema al que se pretende 
dar solución, abordando tanto los aspectos conceptuales del dominio como las implicaciones 
técnicas y funcionales del sistema desde una perspectiva inicial.

Partiendo de las necesidades recogidas en las \hyperref[sec:historias_usuario]{historias de usuario}, 
este capítulo analiza las características del dominio heráldico desde una perspectiva 
técnica y funcional. Para ello, se modelan los elementos relevantes mediante un enfoque 
basado en Domain-Driven Design (DDD), lo que permite establecer una representación 
estructurada de los conceptos clave.

Además, se analizarán los riesgos técnicos asociados y se esbozarán posibles vías de 
evolución futura. Finalmente, se elegirán el lenguaje de programación y las herramientas
que servirán de base para las decisiones de implementación que se describirán en el capítulo 
siguiente.

\section{Modelado del problema}
El modelado del problema se llevará a cabo utilizando un enfoque de Domain-Driven Design (DDD),
una metodología que se basa en reconocer las \textit{entidades}, que son clases encargadas de 
contener lógica de negocio y de gestionar otras estructuras relacionadas, así como los 
\textit{objetos valor}, elementos inmutables cuyo ciclo de vida depende de las entidades que 
los integran.
Se ha elegido DDD porque permite estructurar de forma clara un dominio complejo como el heráldico, 
facilitando que el software refleje fielmente sus conceptos y relaciones.

\subsection{Entidades principales}
A partir del análisis de las historias de usuario, se han identificado las siguientes entidades:

\begin{itemize}
    \item \textbf{Escudo: } Un escudo es la representación heráldica de una persona, linaje o institución.
    Cada escudo tiene un nombre y una descripción (blasón), y está compuesto de:
    \begin{itemize}
        \item \textbf{Campo: } Superficie del escudo donde se sitúan las figuras.
        \item \textbf{Boca: } Perímetro que delimita el escudo.
        \item \textbf{Particiones: } Divisiones del escudo.
        \item \textbf{Esmaltes: } Es el atributo cromático de un campo.
        \item \textbf{Piezas heráldicas: } Suelen tocar casi siempre los bordes del escudo, y suelen ser de
        esmalte diferente al del campo.
        \item \textbf{Muebles o figuras: } Elementos que se sitúan sobre el campo del escudo. Pueden ser
        figuras naturales, artificiales o fantásticas.
        \item \textbf{Adornos exteriores: } Son todos aquellos elementos que no se encuentran dentro del 
        campo del escudo. Los escudos eclesiásticos tienen adornos exteriores específicos.
    \end{itemize}

\end{itemize}