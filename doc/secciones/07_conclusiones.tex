\chapter{Conclusiones y trabajos futuros}

En este capítulo se presentan las conclusiones derivadas del desarrollo del proyecto y las posibles 
líneas de trabajo que podrían continuarlo o ampliarlo en el futuro.

\section{Conclusiones}

El desarrollo de este Trabajo de Fin de Grado ha supuesto una experiencia integral que ha abarcado 
desde la recopilación y normalización de información heráldica hasta el diseño y la implementación 
de una aplicación web funcional. El proyecto ha cumplido su propósito principal: crear una base 
tecnológica que permita analizar, consultar y gestionar información heráldica de forma estructurada 
y accesible.

El sistema desarrollado responde directamente a las dos historias de usuario definidas al inicio 
del proyecto, las cuales guiaron su diseño funcional:

\begin{itemize}
    \item \hyperref[sec:hu1]{\textbf{HU1: Estudiante de Historia del Arte.}}
    El proyecto proporciona una plataforma que permite la búsqueda y consulta de escudos heráldicos 
    a partir de descripciones o atributos, de manera que un usuario interesado, como un estudiante 
    que se encuentra ante un escudo en un edificio o monumento, pueda identificar fácilmente 
    su procedencia y obtener información contextual. Aunque la funcionalidad se cumple, el número 
    de escudos disponibles en la base de datos es todavía limitado, por lo que se prevé su ampliación 
    en futuras versiones.

    \item \hyperref[sec:hu2]{\textbf{HU2: Investigador en heráldica.}} 
    El sistema facilita el análisis, clasificación y filtrado de escudos a través de una base de datos 
    estructurada y consultas textuales flexibles, lo que reduce la dependencia de fuentes analógicas 
    dispersas y mejora la eficiencia en los procesos de estudio y catalogación. Al igual que en el caso 
    anterior, el potencial del sistema aumentará conforme se amplíe el número de registros 
    heráldicos disponibles.
\end{itemize}

De este modo, ambas historias de usuario se han satisfecho mediante el desarrollo de una herramienta 
web accesible y funcional que combina aspectos técnicos y humanísticos.

A lo largo del trabajo se han alcanzado los objetivos definidos inicialmente. Se ha diseñado una base 
de datos coherente con la lógica de la heráldica, se ha implementado una interfaz web con 
\textbf{Flask} y el motor de plantillas \textbf{Jinja2}, utilizando \textbf{Pico.css} para la maquetación 
y el diseño de una interfaz ligera y adaptable, y se ha desarrollado un sistema de búsqueda textual capaz de 
reconocer esmaltes, figuras y otros atributos relevantes. Además, se han aplicado técnicas de 
normalización lingüística para mejorar la calidad de las consultas, teniendo en cuenta variantes y 
sinónimos del vocabulario heráldico.

Durante el desarrollo se han puesto en práctica metodologías ágiles, control de versiones con 
\textbf{Git} y validación del código mediante herramientas automáticas como \textbf{Ruff} o 
\textbf{Textidote}. Estos aspectos han permitido mantener un flujo de trabajo ordenado, 
reproducible y fácilmente ampliable.

En el ámbito personal y académico, este proyecto me ha ayudado a trabajar de una forma mucho más 
organizada y cercana a la realidad del desarrollo profesional. He podido aplicar una metodología ágil 
adaptada al contexto del TFG, lo que me ha permitido dividir el trabajo en tareas claras y ver avances 
reales en cada iteración. Esta manera de trabajar me ha resultado motivadora y útil para mantener el 
ritmo y la coherencia entre lo planificado y lo implementado.

También ha sido una oportunidad para experimentar con distintas maneras de estructurar una aplicación 
real, desde la definición de modelos y vistas hasta la integración de plantillas y la validación del 
código. A lo largo del proceso he aprendido a tomar decisiones de diseño fundamentadas, resolver 
problemas de manera autónoma y consolidar conocimientos técnicos en entornos reales de desarrollo.

En conjunto, este proyecto ha servido para afianzar mi experiencia en el desarrollo web y en la 
organización del trabajo con criterios profesionales, reforzando la relación entre planificación, 
implementación y mantenimiento del software.

\section{Trabajos futuros}

Aunque el sistema actual cumple los objetivos planteados, el proyecto deja abiertas diversas líneas de mejora que podrían abordarse en el futuro. Algunas de las más relevantes son las siguientes:

\begin{itemize}
    \item \textbf{Ampliar la base de datos heráldica:} incorporar más escudos, descripciones y fuentes 
    documentales. Esto permitiría ofrecer resultados más variados y precisos, y cubrir una mayor 
    diversidad geográfica y estilística.
    \item \textbf{Integrar técnicas de inteligencia artificial:} explorar el uso de modelos que asocien 
    imágenes de blasones con sus descripciones o que asistan en la interpretación automática del lenguaje
    heráldico.
    \item \textbf{Mejorar la búsqueda semántica:} refinar la normalización y el tratamiento del lenguaje, 
    incluyendo detección de sinónimos y variantes ortográficas, con el objetivo de que las consultas sean
    más naturales y flexibles.
    \item \textbf{Generar representaciones visuales:} implementar un sistema que permita crear o 
    reconstruir blasones a partir de descripciones textuales, combinando análisis semántico y 
    renderización gráfica.
    \item \textbf{Publicar y desplegar la aplicación:} poner el sistema a disposición de otros usuarios 
    o investigadores a través de un despliegue en la nube, con el fin de facilitar su utilización y 
    fomentar la colaboración.
\end{itemize}

Estas posibles ampliaciones permitirían que el proyecto evolucionara hacia una herramienta más completa y
útil tanto para la comunidad investigadora como para quienes se interesen por la heráldica. Además, 
abrirían la puerta a la aplicación de técnicas de inteligencia artificial en un ámbito en el que aún 
existen pocas soluciones digitales.