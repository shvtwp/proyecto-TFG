\chapter{Introducción}

En este capítulo se presenta la motivación del proyecto, así como la definición
del problema que busca abordar. Se incluyen además las historias de usuario que
justifican el desarrollo de la solución propuesta, los objetivos iniciales y la
estructura del documento.

\section{La heráldica} \label{sec:heraldica}

Antes de hablar de la motivación de este proyecto, conviene definir qué se entiende 
por heráldica. Se trata de la disciplina que estudia los escudos de armas y sus componentes 
simbólicos, estructurales y normativos. A lo largo del tiempo, ha desarrollado un lenguaje 
técnico propio, el blasonado, que permite describir escudos con precisión mediante palabras.  

La digitalización de este conocimiento plantea retos importantes, tanto por la 
complejidad de su terminología como por la diversidad de tradiciones existentes. Estas varían 
según la región geográfica (heráldica española, anglosajona, etc.) o el ámbito de aplicación 
(civil, nobiliario, eclesiástico, etc.), lo que dificulta la creación de modelos informáticos 
universales.

Aunque muchas tradiciones comparten elementos comunes, cada una posee convenciones propias en 
cuanto a formas, vocabulario y reglas compositivas. Por ejemplo, no es lo mismo un escudo inglés
que uno español, ni uno civil que uno eclesiástico. 

Adelanto que este proyecto se centrará específicamente en la heráldica española y eclesiástica, 
atendiendo a las necesidades de los clientes las cuales se expondrán de manera más específica en 
el \hyperref[sec:historias_usuario]{capítulo de planificación}.

\section{Motivación}
Actualmente, el análisis y la consulta de escudos heráldicos presentan barreras
prácticas tanto para quienes se encuentran ocasionalmente con ellos como para 
investigadores especializados. Un estudiante de Historia del Arte, por ejemplo, 
puede hallar escudos en fachadas de edificios históricos y preguntarse qué representará,
sin disponer de una herramienta accesible para su búsqueda e identificación. 
Por otro lado,  un investigador en heráldica debe analizar y clasificar manualmente
los escudos, ya que los medios actuales son en su mayoría analógicos y dispersos, lo 
que convierte esto en un proceso lento y propenso a errores.

Estas situaciones ponen de manifiesto la necesidad de una solución digital que
facilite el acceso, la consulta y el análisis estructurado de escudos heráldicos. 
Este proyecto surge con \textbf{el objetivo de responder a dichas necesidades}, planteando 
una herramienta que permita representar, organizar e interpretar escudos de armas 
de forma más eficiente, moderna y accesible para todos.

\section{Definición del problema}

La falta de una herramienta tecnológica que permita analizar y clasificar escudos de 
armas de manera eficiente impide que estudiantes o investigadores puedan consultar, 
clasificar o comparar escudos de manera eficiente, ya que en la actualidad se requieren
conocimientos especializados y un acceso a fuentes dispersas, muchas de ellas no 
digitalizadas.

Esta situación presenta varias dificultades:

\begin{itemize}
    \item La consulta de escudos requiere revisar múltiples fuentes no centralizadas, 
    en su mayoría analógicas.
    \item No existen bases de datos estandarizadas que permitan buscar escudos por
    características visuales o estructurales.
    \item El análisis se realiza manualmente, lo que ralentiza el proceso y puede
    generar errores.
\end{itemize}

\section{Estructura del proyecto}
El resto del documento se organiza en los siguientes capítulos:

\begin{itemize}
    \item \textbf{Capítulo 2: Estado del arte - }Se analizan trabajos previos relacionados
    con la digitalización de la heráldica y las herramientas existentes, estableciendo
    el contexto teórico del proyecto.
    \item \textbf{Capítulo 3: Planificación - }Se explica la planificación y metodología
    que se emplean en el desarrollo. Se habla de los milestones del proyecto.
    \item \textbf{Capítulo 4: Análisis - }Define qué debe hacer la herramienta y habla
    sobre su diseño y arquitectura.
    \item \textbf{Capítulo 5: Implementación - }Describe cómo se ha desarrollado y validado
    la herramienta. 
    \item \textbf{Capítulo 6: Conclusiones - }Se resumen los resultados obtenidos y se
    discuten posibles mejoras y líneas futuras de investigación.
\end{itemize}

Este proyecto es software libre, y está liberado con la licencia \cite{gplv3}.