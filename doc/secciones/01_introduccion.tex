\chapter{Introducción}

Durante los siglos XI y XII, la evolución del equipo militar hizo prácticamente
imposible reconocer la identidad de un caballero en el campo de batalla, ya que 
cubrían todo su cuerpo y rostro. Para hacer más fácil esta labor de
reconocimiento, comenzaron los caballeros a pintar sobre su escudo ciertas formas
y figuras específicas con la finalidad de distinguirse en las contiendas. 

Estas representaciones que comenzaron siendo personales, pasaron a ser transmitidas
a los descendientes de los primeros poseedores, convirtiéndose en símbolos
hereditarios y emblemas familiares. 

Alrededor de los siglos XIV y XV, desembocado por la desaparición de las justas y 
los torneos, los escudos pasan de ser meros distintivos de identidad a representar
historias de alianza entre linajes y entidades más amplias como casas nobiliarias,
reinos, ciudades y otros colectivos. En definitiva, pasan de ser un símbolo distintivo
a adquirir un enfoque más artístico, que además de distinguir, sirven como una decoración
de palacios y espacios públicos hecha para apreciarse más de cerca.

En este contexto surge formalmente la heráldica, un sistema gráfico que regula el uso 
de estos símbolos y que, etimológicamente, proviene de la palabra \textit{herald}, 
traducida como 'anuncio' o 'llamada'. La heráldica se convierte así en una ciencia-arte
que establece un lenguaje visual propio que permite expresar de manera gráfica el linaje
y la posición de un individuo o colectivo dentro de un contexto social determinado.

Actualmente, la heráldica sigue siendo un campo de estudio relevante en disciplinas
como la historia y la historia del arte, pero presenta obstáculos que dificultan el 
acceso y la comprensión a quien desea introducirse a ella. La falta de una completa
digitalización y estandarización de esta disciplina, y la ausencia de tecnologías
actuales que faciliten su estudio, resulta en barreras para su aprendizaje.

Con la finalidad de hacer menos hostil el aprendizaje de esta ciencia-arte, este
proyecto busca desarrollar una herramienta que modernice y agilice el aprendizaje
de la heráldica, facilitando su comprensión y análisis, aspirando a ofrecer un 
recurso accesible que contribuya a la preservación y divulgación de esta disciplina.

Este proyecto es software libre, y está liberado con la licencia \cite{gplv3}.