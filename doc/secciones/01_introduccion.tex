\chapter{Introducción}

En este capítulo se presenta la motivación del proyecto, así como la definición
del problema que busca abordar. Se incluyen además las historias de usuario que
justifican el desarrollo de la solución propuesta, los objetivos iniciales y la
estructura del documento.

\section{Motivación}
La heráldica es un sistema de representación gráfica codificado que permite
la representación visual de linajes, instituciones y territorios a través de
un conjunto no arbitrario de reglas de diseño y composición. Estas normas
establecidas la hacen un campo apropiado para el análisis estructurado y la
automatización mediante herramientas digitales.

Sin embargo, quienes desean introducirse en esta disciplina se enfrentan a
ciertos obstáculos que dificultan su acceso y comprensión a, debido a que su
documentación se encuentra dispersa y no está completamente digitalizada. Estas 
barreras complican su aprendizaje, aplicación académica y divulgación.

Con la finalidad de satisfacer las necesidades de este grupo de personas y hacer
de esa manera menos hostil el aprendizaje de esta ciencia-arte, este proyecto busca 
desarrollar una herramienta que modernice y agilice el aprendizaje de la heráldica, 
facilitando su comprensión y análisis, aspirando a ofrecer un recurso digital accesible
que contribuya a la preservación y divulgación de esta disciplina.

\section{Definición del problema}

La falta de una herramienta tecnológica que permita analizar y clasificar escudos de 
armas de manera eficiente impide que las personas que estudian o investigan la heráldica
puedan consultar, clasificar o comparar escudos de forma eficiente, ya que en la actualidad
se requieren conocimientos especializados y un acceso a fuentes dispersas, muchas de ellas
no digitalizadas.

Esta situación presenta varias dificultades:

\begin{itemize}
    \item La consulta de escudos requiere revisar múltiples fuentes no centralizadas, 
    en su mayoría analógicas.
    \item No existen bases de datos estandarizadas que permitan buscar escudos por
    características visuales o estructurales.
    \item El análisis se realiza manualmente, lo que ralentiza el proceso y puede
    generar errores.
\end{itemize}

El objetivo de este proyecto será dar solución a los problemas presentados por los
clientes en las siguientes historias de usuario:

\subsection{Historias de usuario}
Con el fin de exponer y comprender mejor las necesidades a las que responde este
proyecto, se presentan los siguientes casos de uso:

\begin{itemize}
    \item \textbf{HU1: Estudiante de Historia del Arte - }Un estudiante de Historia del 
    Arte está paseando por Roma y se encuentra con una obra arquitectónica en cuya fachada
    hay un escudo heráldico. Se pregunta de quién será y qué representa. Sigue caminando 
    y se encuentra con más escudos, lo que le hace pensar en lo oportuna que sería una 
    plataforma que le ayude a identificarlos y obtener información sobre ellos de manera 
    sencilla.
    \item \textbf{HU2: Investigador en heráldica - }Un investigador encuentra tedioso y
    complicado analizar y comprobar escudos en fuentes analógicas, ya que hacerlo manualmente
    de esta manera lo convierte en un proceso lento y propenso a errores.
\end{itemize}

\subsection{Objetivos iniciales}
A partir de las anteriores historias de usuario, se pueden establecer el primer
producto mínimamente viable (o primer milestone):
\begin{enumerate}
    \item \textbf{Milestone 1: Modelo del problema - }Este milestone tiene como objetivo
    crear una base sólida mediante la creación de una representación básica y estructurada
    de los escudos, lo que permitirá almacenar y organizar sus atributos clave.
\end{enumerate}

A medida que se avance en este milestone, se establecerá el siguiente paso de manera iterativa,
ajustándose a los resultados obtenidos y con la finalidad de ofrecer valor continuo al usuario.

\section{Estructura del proyecto}
Es resto del documento se organiza en los siguientes capítulos:

\begin{itemize}
    \item \textbf{Capítulo 2: Estado del arte - }Se analizan trabajos previos relacionados
    con la digitalización de la heráldica y las herramientas existentes, estableciendo
    el contexto teórico del proyecto.
    \item \textbf{Capítulo 3: Planificación - }Se explica la planificación y metodología
    que se emplean en el desarrollo. Se habla de los milestones del proyecto.
    \item \textbf{Capítulo 4: Análisis - }Define qué debe hacer la herramienta y habla
    sobre su diseño y arquitectura.
    \item \textbf{Capítulo 5: Implementación - }Describe cómo se ha desarrollado y validado
    la herramienta. 
    \item \textbf{Capítulo 6: Conclusiones - }Se resumen los resultados obtenidos y se
    discuten posibles mejoras y líneas futuras de investigación.
\end{itemize}

Este proyecto es software libre, y está liberado con la licencia \cite{gplv3}.