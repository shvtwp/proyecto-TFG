\chapter{Introducción}

En este capítulo se presenta la motivación del proyecto, así como la definición
del problema que busca abordar. Se incluyen además las historias de usuario que
justifican el desarrollo de la solución propuesta, los objetivos iniciales y la
estructura del documento.

\section{Motivación}
La heráldica es un sistema de representación gráfica codificado que permite
la representación visual de linajes, instituciones y territorios a través de
un conjunto no arbitrario de reglas de diseño y composición. Estas normas
establecidas la hacen un campo apropiado para el análisis estructurado y la
automatización mediante herramientas digitales.

Sin embargo, actualmente el estudio de esta ciencia-arte presenta ciertos 
obstáculos que dificultan el acceso y la comprensión a quien desea
introducirse en ella, debido a que su documentación se encuentra dispersa y
no está completamente digitalizada. Estas barreras complican su aprendizaje,
aplicación académica y divulgación.

Con la finalidad de hacer menos hostil el aprendizaje de esta ciencia-arte, este
proyecto busca desarrollar una herramienta que modernice y agilice el aprendizaje
de la heráldica, facilitando su comprensión y análisis, aspirando a ofrecer un 
recurso digital accesible que contribuya a la preservación y divulgación de esta 
disciplina.

\section{Definición del problema}
El principal problema que enfrenta el estudio de la heráldica es la falta de una 
herramienta tecnológica que permita analizar y clasificar escudos de armas de manera 
eficiente. En la actualidad, la interpretación y el estudio de los escudos heráldicos 
requieren conocimientos especializados y un acceso a fuentes dispersas, ya que no
existe un sistema automatizado que permita identificar escudos a partir de
su estructura o sus elementos gráficos.

Esta situación presenta varias dificultades:

\begin{itemize}
    \item La consulta de escudos requiere revisar múltiples fuentes dispersas, 
    muchas de ellas no digitalizadas.
    \item No existen bases de datos estandarizadas que permitan buscar escudos por
    características visuales o estructurales.
    \item Los investigadores deben realizar análisis manuales, lo que ralentiza el
    proceso y puede generar errores.
\end{itemize}

El objetivo de este proyecto es desarrollar una base de datos estructurada de escudos
heráldicos junto con herramientas que permitan su consulta y análisis digital. De este
modo, se busca facilitar la investigación y el aprendizaje de la heráldica mediante un
sistema accesible y eficiente.

\subsection{Historias de usuario}
Con el fin de exponer y comprender mejor las necesidades a las que responde este
proyecto, se presentan los siguientes casos de uso:

\begin{itemize}
    \item \textbf{HU0: Tribunal del Trabajo de Fin de Grado - } Como tribunal del Trabajo
    Final de Grado, necesito que esté presente una solución bien justificada, cuyos
    aspectos a valorar son la búsqueda y el tratamiento de la información, la autonomía
    e iniciativa, una planificación clara, la solidez del análisis del problema, una 
    correcta comunicación escrita y una implementación que demuestre calidad técnica,
    de modo que pueda evaluar la viabilidad y el impacto del proyecto.
    \item \textbf{HU1: Estudiante de Historia del Arte - } Un estudiante de Historia del 
    Arte está paseando por Roma y se encuentra con una obra arquitectónica en cuya fachada
    hay un escudo heráldico. Se pregunta de quién será y qué representa. Sigue caminando 
    y se encuentra con más escudos, lo que le hace pensar en lo oportuna que sería una 
    plataforma que le ayude a identificarlos y obtener información sobre ellos de manera 
    sencilla.
    \item \textbf{HU2: Investigador en heráldica - } Un investigador necesita clasificar
    escudos en función de su estructura, colores, figuras, etc. Al no disponer de 
    herramientas digitales que faciliten el proceso, debe hacerlo manualmente, revisando
    manuscritos y otros documentos analógicos, tarea tediosa y propensa a errores.

\end{itemize}

\subsection{Objetivos iniciales}
A partir de las anteriores historias de usuario se plantean los siguientes objetivos
iniciales:

\begin{enumerate}
    \item \textbf{Definir una infraestructura técnica adecuada,} estableciendo las herramientas
    necesarias para el desarrollo del sistema.
    \item \textbf{Diseñar un modelo de representación de la heráldica} que permita clasificar y 
    describir los escudos en función de sus elementos gráficos.
    \item \textbf{Implementar una base de datos} que almacene la información heráldica de manera
    organizada, facilitando la consulta y el análisis.
    \item \textbf{Desarrollar una interfaz de consulta} que permita a los usuarios buscar escudos
    según sus características.
    \item \textbf{Elaborar tests} para comprobar el correcto funcionamiento del sistema.
\end{enumerate}

\section{Estructura del proyecto}
Es resto del documento se organiza en los siguientes capítulos:

\begin{itemize}
    \item \textbf{Capítulo 2: Estado del arte -} Se analizan trabajos previos relacionados
    con la digitalización de la heráldica y las herramientas existentes, estableciendo
    el contexto teórico del proyecto.
    \item \textbf{Capítulo 3: Planificación - } Se explica la planificación y metodología
    que se emplean en el desarrollo. Se habla de los milestones del proyecto.
    \item \textbf{Capítulo 4: Análisis - } Define qué debe hacer la herramienta y habla
    sobre su diseño y arquitectura.
    \item \textbf{Capítulo 5: Implementación - }Describe cómo se ha desarrollado y validado
    la herramienta. 
    \item \textbf{Capítulo 6: Conclusiones - } Se resumen los resultados obtenidos y se
    discuten posibles mejoras y líneas futuras de investigación.
\end{itemize}

Este proyecto es software libre, y está liberado con la licencia \cite{gplv3}.