\chapter{Costes}
En este capítulo se detallan los costes asociados al desarrollo del proyecto,
incluyendo los costes de hardware, software y personal. Aunque no se han generado
gastos económicos directos, se presenta una estimación del valor de los recursos y
del tiempo invertido con el fin de reflejar el esfuerzo real del desarrollo.

\section{Costes de Hardware}

El desarrollo del proyecto se ha realizado utilizando el siguiente hardware:

\begin{table}[H]
  \centering
  \begin{tabular}{|l|l|l|l|l|}
    \hline
    \textbf{Hardware}   & \textbf{Valor} & 
    \href{https://sede.agenciatributaria.gob.es/Sede/ayuda/manuales-videos-folletos/manuales-practicos/folleto-actividades-economicas/3-impuesto-sobre-renta-personas-fisicas/3_5-estimacion-directa-simplificada/3_5_4-tabla-amortizacion-simplificada.html?}{\textbf{Amortización (\%)}} 
    & \textbf{Tiempo de uso} & \textbf{Amortización} \\ \hline
    ASUS ZenBook UX431DA UM431DA & 650 €         & 26\% a 4 años              & 9 meses                & Total: 126,75 €        \\ \hline
  \end{tabular}
  \caption{Tabla de los costes relativos del Hardware al desarrollo.}
  \label{tab:costes_hardware}
\end{table}

\section{Costes de Software de desarrollo}
Durante el desarrollo del proyecto se han empleado herramientas de software libre o cubiertas por 
licencias educativas, por lo que no se han incurrido en costes de software.

\section{Coste de personal}

El Trabajo de Fin de Grado tiene una carga lectiva de 12 créditos ECTS. Según el Sistema Europeo 
de Transferencia de Créditos, cada crédito ECTS equivale a entre 25 y 30 horas de trabajo del 
estudiante, lo que supone una dedicación total de entre 300 y 360 horas.

Para este proyecto, se estima una dedicación total de \textbf{330 horas}, distribuidas a lo largo 
de los diferentes hitos y fases de desarrollo. Este tiempo incluye todas las actividades necesarias 
para completar el trabajo: investigación y análisis de requisitos, diseño de la arquitectura del 
sistema, configuración del entorno de desarrollo, implementación de los distintos módulos, 
realización de pruebas, refactorización del código, redacción de la memoria y preparación de la 
presentación final.

La distribución temporal aproximada de las horas dedicadas es la siguiente:

\begin{table}[H]
\centering
\begin{tabular}{lr}
\toprule
\textbf{Actividad} & \textbf{Horas} \\
\midrule
Investigación y estado del arte & 35 \\
Selección y configuración de herramientas & 45 \\
Diseño de la arquitectura del sistema & 50 \\
Implementación (scraping, modelos, BD, interfaz) & 85 \\
Pruebas y validación & 35 \\
Refactorización y mejoras & 20 \\
Documentación (redacción de la memoria) & 60 \\
\midrule
\textbf{Total} & \textbf{330} \\
\bottomrule
\end{tabular}
\caption{Distribución de horas dedicadas al TFG.}
\label{tab:distribucion-horas}
\end{table}

Considerando un salario medio de un ingeniero junior en España de aproximadamente 20\,000~€ 
brutos anuales, y asumiendo 1.680 horas laborables al año (40 horas semanales durante 42 semanas), 
el coste por hora se sitúa en torno a los 12~€. Por tanto, el coste de personal estimado para 
este proyecto es de:

\[
330 \text{ horas} \times 12~\text{€/hora} = 3.960~\text{€}
\]


\section{Costes de despliegue}

El proyecto se ha ejecutado localmente, sin necesidad de servidores de pago.
En caso de publicación futura, podrían emplearse servicios gratuitos como
\textbf{Vercel}, \textbf{Render} o \textbf{GitHub Pages}, suficientes para el alcance actual del sistema.
En consecuencia, el coste de despliegue es \textbf{0~€}.

\section{Coste total estimado}

\begin{table}[H]
\centering
\begin{tabularx}{0.6\textwidth}{l>{\raggedleft\arraybackslash}X}
\toprule
\textbf{Categoría} & \textbf{Coste (€)} \\
\midrule
Hardware (amortización) & 121,88 \\
Software & 0 \\
Personal & 3\,960 \\
Despliegue & 0 \\
\midrule
\textbf{Total estimado} & \textbf{4\,081,88} \\
\bottomrule
\end{tabularx}
\caption{Resumen de costes del proyecto.}
\end{table}

\section{Conclusión}

Aunque el proyecto no ha supuesto gastos económicos directos,
representa una inversión de aproximadamente \textbf{330 horas de trabajo técnico},
además del uso continuado de recursos materiales y herramientas de desarrollo.
El coste total estimado asciende a \textbf{4\,081,88~€}, lo que refleja el valor del
esfuerzo invertido en su diseño, desarrollo y documentación.